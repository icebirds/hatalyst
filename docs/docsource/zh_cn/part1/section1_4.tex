\section{安装haXe}
接下来,我们就要开始haXe的安装了。

首先,在haXe的官方网站(http://haxe.org/download)下载haXe的安装包,注意选择你所使用的平台。目前官网提供Windows、MAC和Linux的安装包。如果你使用FreeBSD等操作系统,可以按照页面上的指引手动编译haXe。

下载后,Windows平台直接双击下载的安装包文件,按照提示安装即可;Linux平台下,请打开终端,转到安装包所在目录,执行以下命令\footnote{Ubuntu等发行版的软件源中虽然有haXe,但往往版本更新有所延迟,因此建议Linux用户下载安装包安装。}:
\begin{lstlisting}
$sudo ./hxinst-linux
\end{lstlisting}
同样按照提示安装即可。MAC平台做法与上述类似。如果以前安装过haXe,也可以通过上述方法更新。

本书中后面的章节还会使用一些haXe的库,因此在这里需要配置一下haXe的库路径。Windows用户在安装好后就已经配置好了haXe的库路径\footnote{Windows下haXe的默认库目录是C:{\textbackslash}Programs Files{\textbackslash}Motion-Twin{\textbackslash}haxe{\textbackslash}lib。但Windows 7下由于系统增加了大量的访问限制,导致此目录不可用,因此默认的库目录变成了C:{\textbackslash}Motion-Twin{\textbackslash}haxe{\textbackslash}lib,关于此改动的详情请参见http://haxe.org/download。},可以跳过这一步\footnote{Windows一般不必配置haxelib的库路径,但由于Windows下库路径默认被放在系统分区中,因此一旦重新安装操作系统,就需要重新安装和配置库。而Linux下由于可以把库全部安装在用户目录中,所以重新安装后只需要将库路径重新设置即可。如Windows用户希望重装系统后比较方便,也可配置库路径。}。

Linux用户和MAC用户可以在终端下运行以下命令进行haXe库路径的配置。

\begin{lstlisting}
$haxelib setup
Please enter haxelib repository path with write access 
Path: ~/haxelib
\end{lstlisting}

然后输入一个准备好的空目录路径即可。如上面的 ~/haxelib。
