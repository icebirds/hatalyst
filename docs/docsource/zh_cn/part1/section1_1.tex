\section{什么是haXe}

haXe(读做\/heks\/\footnote{官方网站给出的读音读法为“hex”,但这个注音方式在不同的语言背景下差异极大,部分中文用户甚至将其读为“哈希”、“海西”这样的读音,为了避免读音歧义,这里采用的是国际音标注音的方式,中文读者可以近似地认为其读音接近“嗨克斯”。})是一款开源的多平台编程语言(作者注:个人觉得称其为多输出编程语言更合适一些),不同于传统的编程语言,haXe并不生成自己独 立的字节码(如Java),也不直接生成本地代码\footnote{如有需要,开发人员可以给haXe增加本地码编译的目标,但目前haXe并没有这个功能。}(如 C++,不过未来有可能会被加入直接生成本地代码的功能),而是编译生成现有的字节码或程序语言源码。其所生成的每个种类的源码或字节码被称为编译目标。

目前haXe可以编译为以下目标:

\begin{enumerate}
\item {{\bf Javascript}:haXe可以编译为一个单独的.js文件,并嵌入html中执行。由于在编译时会解决所有的平台相关的问题,因此可以通过统一的接口方便地访问浏览器的DOM树。}
\item {{\bf Flash}:你可以把haXe代码编译为.swf文件,通过对AS 2和AS 3的API的分别支持,haXe可以编译输出针对Flash player 6-10.2的任意版本swf程序。haXe通过在编译时对AVM字节码的优化以及增加了一些更底层的编译支持,haXe生成的.swf文件可以获得比 Adobe Flash平台编译程序略高的执行效率以及更好的代码体验。 }
\item {{\bf NekoVM}:haXe可以编译为NekoVM字节码(NekoVM是一种类JVM的轻量级虚拟机,采用Neko编程语言编写程序并生成字节码),可以用于web服务器或socket服务器的开发,亦可通过扩展用于桌面开发。}
\item {{\bf PHP}:haXe代码可以编译为.php文件,其编译结果接近一个超轻量级的框架。通过对外部代码库的支持,可以很方便地将现有的php库用于你的haXe项目。}
\item {{\bf C++}:haXe代码可以生成C++代码,并调用不同的C++编译器编译为本地程序。这个编译目标在希望将你的程序生成某些本地程序(如Iphone应用)时非常有用。}
\item {在本书编写时,有消息称C\#和Java两种编译目标也即将发布。}
\end{enumerate}
